% Generated by Sphinx.
\def\sphinxdocclass{report}
\documentclass[letterpaper,10pt,english]{sphinxmanual}
\usepackage[utf8]{inputenc}
\DeclareUnicodeCharacter{00A0}{\nobreakspace}
\usepackage{cmap}
\usepackage[T1]{fontenc}
\usepackage{babel}
\usepackage{times}
\usepackage[Bjarne]{fncychap}
\usepackage{longtable}
\usepackage{sphinx}
\usepackage{multirow}
\usepackage{eqparbox}
\usepackage{amsfonts}

\addto\captionsenglish{\renewcommand{\figurename}{Fig. }}
\addto\captionsenglish{\renewcommand{\tablename}{Table }}
\SetupFloatingEnvironment{literal-block}{name=Listing }



\title{dessn Documentation}
\date{March 01, 2016}
\release{0.0.1}
\author{dessn}
\newcommand{\sphinxlogo}{}
\renewcommand{\releasename}{Release}
\setcounter{tocdepth}{3}
\makeindex

\makeatletter
\def\PYG@reset{\let\PYG@it=\relax \let\PYG@bf=\relax%
    \let\PYG@ul=\relax \let\PYG@tc=\relax%
    \let\PYG@bc=\relax \let\PYG@ff=\relax}
\def\PYG@tok#1{\csname PYG@tok@#1\endcsname}
\def\PYG@toks#1+{\ifx\relax#1\empty\else%
    \PYG@tok{#1}\expandafter\PYG@toks\fi}
\def\PYG@do#1{\PYG@bc{\PYG@tc{\PYG@ul{%
    \PYG@it{\PYG@bf{\PYG@ff{#1}}}}}}}
\def\PYG#1#2{\PYG@reset\PYG@toks#1+\relax+\PYG@do{#2}}

\expandafter\def\csname PYG@tok@gd\endcsname{\def\PYG@tc##1{\textcolor[rgb]{0.63,0.00,0.00}{##1}}}
\expandafter\def\csname PYG@tok@gu\endcsname{\let\PYG@bf=\textbf\def\PYG@tc##1{\textcolor[rgb]{0.50,0.00,0.50}{##1}}}
\expandafter\def\csname PYG@tok@gt\endcsname{\def\PYG@tc##1{\textcolor[rgb]{0.00,0.27,0.87}{##1}}}
\expandafter\def\csname PYG@tok@gs\endcsname{\let\PYG@bf=\textbf}
\expandafter\def\csname PYG@tok@gr\endcsname{\def\PYG@tc##1{\textcolor[rgb]{1.00,0.00,0.00}{##1}}}
\expandafter\def\csname PYG@tok@cm\endcsname{\let\PYG@it=\textit\def\PYG@tc##1{\textcolor[rgb]{0.25,0.50,0.56}{##1}}}
\expandafter\def\csname PYG@tok@vg\endcsname{\def\PYG@tc##1{\textcolor[rgb]{0.73,0.38,0.84}{##1}}}
\expandafter\def\csname PYG@tok@vi\endcsname{\def\PYG@tc##1{\textcolor[rgb]{0.73,0.38,0.84}{##1}}}
\expandafter\def\csname PYG@tok@mh\endcsname{\def\PYG@tc##1{\textcolor[rgb]{0.13,0.50,0.31}{##1}}}
\expandafter\def\csname PYG@tok@cs\endcsname{\def\PYG@tc##1{\textcolor[rgb]{0.25,0.50,0.56}{##1}}\def\PYG@bc##1{\setlength{\fboxsep}{0pt}\colorbox[rgb]{1.00,0.94,0.94}{\strut ##1}}}
\expandafter\def\csname PYG@tok@ge\endcsname{\let\PYG@it=\textit}
\expandafter\def\csname PYG@tok@vc\endcsname{\def\PYG@tc##1{\textcolor[rgb]{0.73,0.38,0.84}{##1}}}
\expandafter\def\csname PYG@tok@il\endcsname{\def\PYG@tc##1{\textcolor[rgb]{0.13,0.50,0.31}{##1}}}
\expandafter\def\csname PYG@tok@go\endcsname{\def\PYG@tc##1{\textcolor[rgb]{0.20,0.20,0.20}{##1}}}
\expandafter\def\csname PYG@tok@cp\endcsname{\def\PYG@tc##1{\textcolor[rgb]{0.00,0.44,0.13}{##1}}}
\expandafter\def\csname PYG@tok@gi\endcsname{\def\PYG@tc##1{\textcolor[rgb]{0.00,0.63,0.00}{##1}}}
\expandafter\def\csname PYG@tok@gh\endcsname{\let\PYG@bf=\textbf\def\PYG@tc##1{\textcolor[rgb]{0.00,0.00,0.50}{##1}}}
\expandafter\def\csname PYG@tok@ni\endcsname{\let\PYG@bf=\textbf\def\PYG@tc##1{\textcolor[rgb]{0.84,0.33,0.22}{##1}}}
\expandafter\def\csname PYG@tok@nl\endcsname{\let\PYG@bf=\textbf\def\PYG@tc##1{\textcolor[rgb]{0.00,0.13,0.44}{##1}}}
\expandafter\def\csname PYG@tok@nn\endcsname{\let\PYG@bf=\textbf\def\PYG@tc##1{\textcolor[rgb]{0.05,0.52,0.71}{##1}}}
\expandafter\def\csname PYG@tok@no\endcsname{\def\PYG@tc##1{\textcolor[rgb]{0.38,0.68,0.84}{##1}}}
\expandafter\def\csname PYG@tok@na\endcsname{\def\PYG@tc##1{\textcolor[rgb]{0.25,0.44,0.63}{##1}}}
\expandafter\def\csname PYG@tok@nb\endcsname{\def\PYG@tc##1{\textcolor[rgb]{0.00,0.44,0.13}{##1}}}
\expandafter\def\csname PYG@tok@nc\endcsname{\let\PYG@bf=\textbf\def\PYG@tc##1{\textcolor[rgb]{0.05,0.52,0.71}{##1}}}
\expandafter\def\csname PYG@tok@nd\endcsname{\let\PYG@bf=\textbf\def\PYG@tc##1{\textcolor[rgb]{0.33,0.33,0.33}{##1}}}
\expandafter\def\csname PYG@tok@ne\endcsname{\def\PYG@tc##1{\textcolor[rgb]{0.00,0.44,0.13}{##1}}}
\expandafter\def\csname PYG@tok@nf\endcsname{\def\PYG@tc##1{\textcolor[rgb]{0.02,0.16,0.49}{##1}}}
\expandafter\def\csname PYG@tok@si\endcsname{\let\PYG@it=\textit\def\PYG@tc##1{\textcolor[rgb]{0.44,0.63,0.82}{##1}}}
\expandafter\def\csname PYG@tok@s2\endcsname{\def\PYG@tc##1{\textcolor[rgb]{0.25,0.44,0.63}{##1}}}
\expandafter\def\csname PYG@tok@nt\endcsname{\let\PYG@bf=\textbf\def\PYG@tc##1{\textcolor[rgb]{0.02,0.16,0.45}{##1}}}
\expandafter\def\csname PYG@tok@nv\endcsname{\def\PYG@tc##1{\textcolor[rgb]{0.73,0.38,0.84}{##1}}}
\expandafter\def\csname PYG@tok@s1\endcsname{\def\PYG@tc##1{\textcolor[rgb]{0.25,0.44,0.63}{##1}}}
\expandafter\def\csname PYG@tok@ch\endcsname{\let\PYG@it=\textit\def\PYG@tc##1{\textcolor[rgb]{0.25,0.50,0.56}{##1}}}
\expandafter\def\csname PYG@tok@m\endcsname{\def\PYG@tc##1{\textcolor[rgb]{0.13,0.50,0.31}{##1}}}
\expandafter\def\csname PYG@tok@gp\endcsname{\let\PYG@bf=\textbf\def\PYG@tc##1{\textcolor[rgb]{0.78,0.36,0.04}{##1}}}
\expandafter\def\csname PYG@tok@sh\endcsname{\def\PYG@tc##1{\textcolor[rgb]{0.25,0.44,0.63}{##1}}}
\expandafter\def\csname PYG@tok@ow\endcsname{\let\PYG@bf=\textbf\def\PYG@tc##1{\textcolor[rgb]{0.00,0.44,0.13}{##1}}}
\expandafter\def\csname PYG@tok@sx\endcsname{\def\PYG@tc##1{\textcolor[rgb]{0.78,0.36,0.04}{##1}}}
\expandafter\def\csname PYG@tok@bp\endcsname{\def\PYG@tc##1{\textcolor[rgb]{0.00,0.44,0.13}{##1}}}
\expandafter\def\csname PYG@tok@c1\endcsname{\let\PYG@it=\textit\def\PYG@tc##1{\textcolor[rgb]{0.25,0.50,0.56}{##1}}}
\expandafter\def\csname PYG@tok@o\endcsname{\def\PYG@tc##1{\textcolor[rgb]{0.40,0.40,0.40}{##1}}}
\expandafter\def\csname PYG@tok@kc\endcsname{\let\PYG@bf=\textbf\def\PYG@tc##1{\textcolor[rgb]{0.00,0.44,0.13}{##1}}}
\expandafter\def\csname PYG@tok@c\endcsname{\let\PYG@it=\textit\def\PYG@tc##1{\textcolor[rgb]{0.25,0.50,0.56}{##1}}}
\expandafter\def\csname PYG@tok@mf\endcsname{\def\PYG@tc##1{\textcolor[rgb]{0.13,0.50,0.31}{##1}}}
\expandafter\def\csname PYG@tok@err\endcsname{\def\PYG@bc##1{\setlength{\fboxsep}{0pt}\fcolorbox[rgb]{1.00,0.00,0.00}{1,1,1}{\strut ##1}}}
\expandafter\def\csname PYG@tok@mb\endcsname{\def\PYG@tc##1{\textcolor[rgb]{0.13,0.50,0.31}{##1}}}
\expandafter\def\csname PYG@tok@ss\endcsname{\def\PYG@tc##1{\textcolor[rgb]{0.32,0.47,0.09}{##1}}}
\expandafter\def\csname PYG@tok@sr\endcsname{\def\PYG@tc##1{\textcolor[rgb]{0.14,0.33,0.53}{##1}}}
\expandafter\def\csname PYG@tok@mo\endcsname{\def\PYG@tc##1{\textcolor[rgb]{0.13,0.50,0.31}{##1}}}
\expandafter\def\csname PYG@tok@kd\endcsname{\let\PYG@bf=\textbf\def\PYG@tc##1{\textcolor[rgb]{0.00,0.44,0.13}{##1}}}
\expandafter\def\csname PYG@tok@mi\endcsname{\def\PYG@tc##1{\textcolor[rgb]{0.13,0.50,0.31}{##1}}}
\expandafter\def\csname PYG@tok@kn\endcsname{\let\PYG@bf=\textbf\def\PYG@tc##1{\textcolor[rgb]{0.00,0.44,0.13}{##1}}}
\expandafter\def\csname PYG@tok@cpf\endcsname{\let\PYG@it=\textit\def\PYG@tc##1{\textcolor[rgb]{0.25,0.50,0.56}{##1}}}
\expandafter\def\csname PYG@tok@kr\endcsname{\let\PYG@bf=\textbf\def\PYG@tc##1{\textcolor[rgb]{0.00,0.44,0.13}{##1}}}
\expandafter\def\csname PYG@tok@s\endcsname{\def\PYG@tc##1{\textcolor[rgb]{0.25,0.44,0.63}{##1}}}
\expandafter\def\csname PYG@tok@kp\endcsname{\def\PYG@tc##1{\textcolor[rgb]{0.00,0.44,0.13}{##1}}}
\expandafter\def\csname PYG@tok@w\endcsname{\def\PYG@tc##1{\textcolor[rgb]{0.73,0.73,0.73}{##1}}}
\expandafter\def\csname PYG@tok@kt\endcsname{\def\PYG@tc##1{\textcolor[rgb]{0.56,0.13,0.00}{##1}}}
\expandafter\def\csname PYG@tok@sc\endcsname{\def\PYG@tc##1{\textcolor[rgb]{0.25,0.44,0.63}{##1}}}
\expandafter\def\csname PYG@tok@sb\endcsname{\def\PYG@tc##1{\textcolor[rgb]{0.25,0.44,0.63}{##1}}}
\expandafter\def\csname PYG@tok@k\endcsname{\let\PYG@bf=\textbf\def\PYG@tc##1{\textcolor[rgb]{0.00,0.44,0.13}{##1}}}
\expandafter\def\csname PYG@tok@se\endcsname{\let\PYG@bf=\textbf\def\PYG@tc##1{\textcolor[rgb]{0.25,0.44,0.63}{##1}}}
\expandafter\def\csname PYG@tok@sd\endcsname{\let\PYG@it=\textit\def\PYG@tc##1{\textcolor[rgb]{0.25,0.44,0.63}{##1}}}

\def\PYGZbs{\char`\\}
\def\PYGZus{\char`\_}
\def\PYGZob{\char`\{}
\def\PYGZcb{\char`\}}
\def\PYGZca{\char`\^}
\def\PYGZam{\char`\&}
\def\PYGZlt{\char`\<}
\def\PYGZgt{\char`\>}
\def\PYGZsh{\char`\#}
\def\PYGZpc{\char`\%}
\def\PYGZdl{\char`\$}
\def\PYGZhy{\char`\-}
\def\PYGZsq{\char`\'}
\def\PYGZdq{\char`\"}
\def\PYGZti{\char`\~}
% for compatibility with earlier versions
\def\PYGZat{@}
\def\PYGZlb{[}
\def\PYGZrb{]}
\makeatother

\renewcommand\PYGZsq{\textquotesingle}

\begin{document}

\maketitle
\tableofcontents
\phantomsection\label{modules::doc}



\chapter{dessn package}
\label{dessn:dessn}\label{dessn::doc}\label{dessn:dessn-package}

\section{Subpackages}
\label{dessn:subpackages}

\subsection{dessn.entry package}
\label{dessn.entry:dessn-entry-package}\label{dessn.entry::doc}

\subsubsection{Submodules}
\label{dessn.entry:submodules}

\subsubsection{dessn.entry.sim module}
\label{dessn.entry:dessn-entry-sim-module}\label{dessn.entry:module-dessn.entry.sim}\index{dessn.entry.sim (module)}

\subsubsection{Module contents}
\label{dessn.entry:module-dessn.entry}\label{dessn.entry:module-contents}\index{dessn.entry (module)}

\subsection{dessn.model package}
\label{dessn.model:dessn-model-package}\label{dessn.model::doc}

\subsubsection{Subpackages}
\label{dessn.model:subpackages}

\paragraph{dessn.model.nodes package}
\label{dessn.model.nodes:dessn-model-nodes-package}\label{dessn.model.nodes::doc}

\subparagraph{Submodules}
\label{dessn.model.nodes:submodules}

\subparagraph{dessn.model.nodes.cosmology module}
\label{dessn.model.nodes:dessn-model-nodes-cosmology-module}\label{dessn.model.nodes:module-dessn.model.nodes.cosmology}\index{dessn.model.nodes.cosmology (module)}\index{Cosmology (class in dessn.model.nodes.cosmology)}

\begin{fulllineitems}
\phantomsection\label{dessn.model.nodes:dessn.model.nodes.cosmology.Cosmology}\pysigline{\strong{class }\code{dessn.model.nodes.cosmology.}\bfcode{Cosmology}}
Bases: {\hyperref[dessn.model:dessn.model.node.Node]{\emph{\code{dessn.model.node.Node}}}}
\index{get\_name() (dessn.model.nodes.cosmology.Cosmology method)}

\begin{fulllineitems}
\phantomsection\label{dessn.model.nodes:dessn.model.nodes.cosmology.Cosmology.get_name}\pysiglinewithargsret{\bfcode{get\_name}}{}{}
\end{fulllineitems}


\end{fulllineitems}

\index{FlatWCDM (class in dessn.model.nodes.cosmology)}

\begin{fulllineitems}
\phantomsection\label{dessn.model.nodes:dessn.model.nodes.cosmology.FlatWCDM}\pysigline{\strong{class }\code{dessn.model.nodes.cosmology.}\bfcode{FlatWCDM}}
Bases: {\hyperref[dessn.model.nodes:dessn.model.nodes.cosmology.Cosmology]{\emph{\code{dessn.model.nodes.cosmology.Cosmology}}}}

\end{fulllineitems}



\subparagraph{dessn.model.nodes.supernova module}
\label{dessn.model.nodes:dessn-model-nodes-supernova-module}\label{dessn.model.nodes:module-dessn.model.nodes.supernova}\index{dessn.model.nodes.supernova (module)}\index{Supernova (class in dessn.model.nodes.supernova)}

\begin{fulllineitems}
\phantomsection\label{dessn.model.nodes:dessn.model.nodes.supernova.Supernova}\pysigline{\strong{class }\code{dessn.model.nodes.supernova.}\bfcode{Supernova}}
Bases: {\hyperref[dessn.model:dessn.model.node.Node]{\emph{\code{dessn.model.node.Node}}}}

Abstract supernova which others must implement
\index{type (dessn.model.nodes.supernova.Supernova attribute)}

\begin{fulllineitems}
\phantomsection\label{dessn.model.nodes:dessn.model.nodes.supernova.Supernova.type}\pysigline{\bfcode{type}}
\end{fulllineitems}


\end{fulllineitems}

\index{SupernovaIa (class in dessn.model.nodes.supernova)}

\begin{fulllineitems}
\phantomsection\label{dessn.model.nodes:dessn.model.nodes.supernova.SupernovaIa}\pysiglinewithargsret{\strong{class }\code{dessn.model.nodes.supernova.}\bfcode{SupernovaIa}}{\emph{log\_luminosity}, \emph{sigma\_luminosity}}{}
Bases: \code{object}

Models a type Ia supernova

Models the luminosity distribution of type Ia supernovas statically (without
internal parameters).
\begin{gather}
\begin{split}P(L) \sim N(\end{split}\notag
\end{gather}\begin{quote}\begin{description}
\item[{Parameters}] \leavevmode
\textbf{log\_luminosity} : float
\begin{quote}

Represented by the variable \(\mu\)
\end{quote}

\textbf{sigma\_luminosity} : float
\begin{quote}

Represented by the variable \(\sigma\)
\end{quote}

\end{description}\end{quote}
\index{get\_luminosity\_prob() (dessn.model.nodes.supernova.SupernovaIa method)}

\begin{fulllineitems}
\phantomsection\label{dessn.model.nodes:dessn.model.nodes.supernova.SupernovaIa.get_luminosity_prob}\pysiglinewithargsret{\bfcode{get\_luminosity\_prob}}{\emph{log\_luminosity}}{}
\end{fulllineitems}

\index{type() (dessn.model.nodes.supernova.SupernovaIa method)}

\begin{fulllineitems}
\phantomsection\label{dessn.model.nodes:dessn.model.nodes.supernova.SupernovaIa.type}\pysiglinewithargsret{\bfcode{type}}{}{}
\end{fulllineitems}


\end{fulllineitems}



\subparagraph{dessn.model.nodes.typeProb module}
\label{dessn.model.nodes:dessn-model-nodes-typeprob-module}\label{dessn.model.nodes:module-dessn.model.nodes.typeProb}\index{dessn.model.nodes.typeProb (module)}\index{TypeProbability (class in dessn.model.nodes.typeProb)}

\begin{fulllineitems}
\phantomsection\label{dessn.model.nodes:dessn.model.nodes.typeProb.TypeProbability}\pysigline{\strong{class }\code{dessn.model.nodes.typeProb.}\bfcode{TypeProbability}}
Bases: {\hyperref[dessn.model:dessn.model.node.Node]{\emph{\code{dessn.model.node.Node}}}}

Abstract type probability node, from which all implementations should inherit
\index{get\_name() (dessn.model.nodes.typeProb.TypeProbability method)}

\begin{fulllineitems}
\phantomsection\label{dessn.model.nodes:dessn.model.nodes.typeProb.TypeProbability.get_name}\pysiglinewithargsret{\bfcode{get\_name}}{}{}
\end{fulllineitems}


\end{fulllineitems}

\index{TypeProbabilitySimple (class in dessn.model.nodes.typeProb)}

\begin{fulllineitems}
\phantomsection\label{dessn.model.nodes:dessn.model.nodes.typeProb.TypeProbabilitySimple}\pysiglinewithargsret{\strong{class }\code{dessn.model.nodes.typeProb.}\bfcode{TypeProbabilitySimple}}{\emph{relative\_rate=0.333}}{}
Bases: {\hyperref[dessn.model.nodes:dessn.model.nodes.typeProb.TypeProbability]{\emph{\code{dessn.model.nodes.typeProb.TypeProbability}}}}

The Type probability node.

Takes \emph{some information} to determine the probability of
of the object in question being a type of supernova.
\begin{quote}\begin{description}
\item[{Parameters}] \leavevmode
\textbf{relative\_rate} : Optional{[}str{]}
\begin{quote}

Relative rate of SnIa / SnII
\end{quote}

\end{description}\end{quote}

\end{fulllineitems}

\index{Types (class in dessn.model.nodes.typeProb)}

\begin{fulllineitems}
\phantomsection\label{dessn.model.nodes:dessn.model.nodes.typeProb.Types}\pysigline{\strong{class }\code{dessn.model.nodes.typeProb.}\bfcode{Types}}
Bases: \code{enum.Enum}

Possible target types
\index{snII (dessn.model.nodes.typeProb.Types attribute)}

\begin{fulllineitems}
\phantomsection\label{dessn.model.nodes:dessn.model.nodes.typeProb.Types.snII}\pysigline{\bfcode{snII}\strong{ = \textless{}Types.snII: 2\textgreater{}}}
\end{fulllineitems}

\index{snIa (dessn.model.nodes.typeProb.Types attribute)}

\begin{fulllineitems}
\phantomsection\label{dessn.model.nodes:dessn.model.nodes.typeProb.Types.snIa}\pysigline{\bfcode{snIa}\strong{ = \textless{}Types.snIa: 1\textgreater{}}}
\end{fulllineitems}


\end{fulllineitems}



\subparagraph{Module contents}
\label{dessn.model.nodes:module-contents}\label{dessn.model.nodes:module-dessn.model.nodes}\index{dessn.model.nodes (module)}

\subsubsection{Submodules}
\label{dessn.model:submodules}

\subsubsection{dessn.model.abstracts module}
\label{dessn.model:module-dessn.model.abstracts}\label{dessn.model:dessn-model-abstracts-module}\index{dessn.model.abstracts (module)}

\subsubsection{dessn.model.model module}
\label{dessn.model:dessn-model-model-module}\label{dessn.model:module-dessn.model.model}\index{dessn.model.model (module)}\index{Model (class in dessn.model.model)}

\begin{fulllineitems}
\phantomsection\label{dessn.model:dessn.model.model.Model}\pysigline{\strong{class }\code{dessn.model.model.}\bfcode{Model}}
Bases: \code{object}

\end{fulllineitems}



\subsubsection{dessn.model.node module}
\label{dessn.model:module-dessn.model.node}\label{dessn.model:dessn-model-node-module}\index{dessn.model.node (module)}\index{Node (class in dessn.model.node)}

\begin{fulllineitems}
\phantomsection\label{dessn.model:dessn.model.node.Node}\pysigline{\strong{class }\code{dessn.model.node.}\bfcode{Node}}
Bases: \code{object}
\index{add\_dependency() (dessn.model.node.Node method)}

\begin{fulllineitems}
\phantomsection\label{dessn.model:dessn.model.node.Node.add_dependency}\pysiglinewithargsret{\bfcode{add\_dependency}}{\emph{dependency\_class}}{}
\end{fulllineitems}

\index{get\_dependencies() (dessn.model.node.Node method)}

\begin{fulllineitems}
\phantomsection\label{dessn.model:dessn.model.node.Node.get_dependencies}\pysiglinewithargsret{\bfcode{get\_dependencies}}{}{}
\end{fulllineitems}

\index{get\_name() (dessn.model.node.Node method)}

\begin{fulllineitems}
\phantomsection\label{dessn.model:dessn.model.node.Node.get_name}\pysiglinewithargsret{\bfcode{get\_name}}{}{}
\end{fulllineitems}


\end{fulllineitems}



\subsubsection{Module contents}
\label{dessn.model:module-dessn.model}\label{dessn.model:module-contents}\index{dessn.model (module)}

\subsection{dessn.simple package}
\label{dessn.simple::doc}\label{dessn.simple:dessn-simple-package}

\subsubsection{Submodules}
\label{dessn.simple:submodules}

\subsubsection{dessn.simple.exampleIntegral module}
\label{dessn.simple:dessn-simple-exampleintegral-module}\label{dessn.simple:module-dessn.simple.exampleIntegral}\index{dessn.simple.exampleIntegral (module)}\index{ExampleIntegral (class in dessn.simple.exampleIntegral)}

\begin{fulllineitems}
\phantomsection\label{dessn.simple:dessn.simple.exampleIntegral.ExampleIntegral}\pysiglinewithargsret{\strong{class }\code{dessn.simple.exampleIntegral.}\bfcode{ExampleIntegral}}{\emph{n=900}, \emph{theta\_1=100.0}, \emph{theta\_2=30.0}}{}
Bases: \code{object}

An example implementation using integration over a latent parameter.

Let us assume that we are observing supernova that a drawn from an underlying
supernova distribution parameterised by \(\theta\),
where the supernova itself simply a luminosity \(L\). We measure the luminosity
of multiple supernovas, giving us an array of measurements \(D\). If we wish to recover
the underlying distribution of supernovas from our measurements, we wish to find \(P(\theta|D)\),
which is given by
\begin{gather}
\begin{split}P(\theta|D) \propto P(D|\theta)P(\theta)\end{split}\notag
\end{gather}
Note that in the above equation, we realise that \(P(D|L) = \prod_{i=1}^N P(D_i|L_i)\) as our measurements are
independent. The likelihood \(P(D|\theta)\) is given by
\begin{gather}
\begin{split}P(D|\theta) =  \prod_{i=1}^N  \int_{-\infty}^\infty P(D_i|L_i) P(L_i|\theta) dL_i\end{split}\notag
\end{gather}
We now have two distributions to characterise. Let us assume both are gaussian, that is
our observed luminosity \(x_i\) has gaussian error \(\sigma_i\) from the actual supernova
luminosity, and the supernova luminosity is drawn from an underlying gaussian distribution
parameterised by \(\theta\).
\begin{quote}
\begin{gather}
\begin{split}P(D_i|L_i) = \frac{1}{\sqrt{2\pi}\sigma_i}\exp\left(-\frac{(x_i-L_i)^2}{\sigma_i^2}\right)\end{split}\notag\\\begin{split}P(L_i|\theta) = \frac{1}{\sqrt{2\pi}\theta_2}\exp\left(-\frac{(L_i-\theta_1)^2}{\theta_2^2}\right)\end{split}\notag
\end{gather}\end{quote}

This gives us a likelihood of
\begin{gather}
\begin{split}P(D|\theta) = \prod_{i=1}^N  \frac{1}{2\pi \theta_2 \sigma_i}  \int_{-\infty}^\infty
\exp\left(-\frac{(x_i-L_i)^2}{\sigma_i^2} -\frac{(L_i-\theta_1)^2}{\theta_2^2} \right) dL_i\end{split}\notag
\end{gather}
Working in log space for as much as possible will assist in numerical precision, so we can rewrite this as
\begin{gather}
\begin{split}\log\left(P(D|\theta)\right) =  \sum_{i=1}^N  \left[
        \log\left( \int_{-\infty}^\infty \exp\left(-\frac{(x_i-L_i)^2}{\sigma_i^2} -
\frac{(L_i-\theta_1)^2}{\theta_2^2} \right) dL_i \right) -\log(2\pi\theta_2\sigma_i) \right]\end{split}\notag
\end{gather}
Creating this class will set up observations from an underlying distribution.
Invoke \code{emcee} by calling the object. Notice that performing the marginalisation over
\(dL_i\) requires computing \(n\) integrals for each step in the MCMC.
\begin{quote}\begin{description}
\item[{Parameters}] \leavevmode
\textbf{n} : int, optional
\begin{quote}

The number of supernova to `observe'
\end{quote}

\textbf{theta\_1} : float, optional
\begin{quote}

The mean of the underlying supernova luminosity distribution
\end{quote}

\textbf{theta\_2} : float, optional
\begin{quote}

The standard deviation of the underlying supernova luminosity distribution
\end{quote}

\end{description}\end{quote}
\index{do\_emcee() (dessn.simple.exampleIntegral.ExampleIntegral method)}

\begin{fulllineitems}
\phantomsection\label{dessn.simple:dessn.simple.exampleIntegral.ExampleIntegral.do_emcee}\pysiglinewithargsret{\bfcode{do\_emcee}}{\emph{nwalkers=20}, \emph{nburn=500}, \emph{nsteps=3000}}{}
Run the \emph{emcee} chain and produce a corner plot.

Saves a png image of the corner plot to plots/exampleIntegration.png.
\begin{quote}\begin{description}
\item[{Parameters}] \leavevmode
\textbf{nwalkers} : int, optional
\begin{quote}

The number of walkers to use. Minimum of four.
\end{quote}

\textbf{nburn} : int, optional
\begin{quote}

The burn in period of the chains.
\end{quote}

\textbf{nsteps} : int, optional
\begin{quote}

The number of steps to run
\end{quote}

\end{description}\end{quote}

\end{fulllineitems}

\index{get\_likelihood() (dessn.simple.exampleIntegral.ExampleIntegral method)}

\begin{fulllineitems}
\phantomsection\label{dessn.simple:dessn.simple.exampleIntegral.ExampleIntegral.get_likelihood}\pysiglinewithargsret{\bfcode{get\_likelihood}}{\emph{theta}, \emph{data}, \emph{error}}{}
Gets the log likelihood given the supplied input parameters.
\begin{quote}\begin{description}
\item[{Parameters}] \leavevmode
\textbf{theta} : array of size 2
\begin{quote}

An array representing \([\theta_1,\theta_2]\)
\end{quote}

\textbf{data} : array of length \emph{n}
\begin{quote}

An array of observed luminosities
\end{quote}

\textbf{error} : array of length \emph{n}
\begin{quote}

An array of observed luminosity errors
\end{quote}

\item[{Returns}] \leavevmode
float
\begin{quote}

the log likelihood probability
\end{quote}

\end{description}\end{quote}

\end{fulllineitems}

\index{get\_posterior() (dessn.simple.exampleIntegral.ExampleIntegral method)}

\begin{fulllineitems}
\phantomsection\label{dessn.simple:dessn.simple.exampleIntegral.ExampleIntegral.get_posterior}\pysiglinewithargsret{\bfcode{get\_posterior}}{\emph{theta}, \emph{data}, \emph{error}}{}
Gives the log posterior probability given the supplied input parameters.
\begin{quote}\begin{description}
\item[{Parameters}] \leavevmode
\textbf{theta} : array of size 2
\begin{quote}

An array representing \([\theta_1,\theta_2]\)
\end{quote}

\textbf{data} : array of length \emph{n}
\begin{quote}

An array of observed luminosities
\end{quote}

\textbf{error} : array of length \emph{n}
\begin{quote}

An array of observed luminosity errors
\end{quote}

\item[{Returns}] \leavevmode
float
\begin{quote}

the log posterior probability
\end{quote}

\end{description}\end{quote}

\end{fulllineitems}

\index{get\_prior() (dessn.simple.exampleIntegral.ExampleIntegral method)}

\begin{fulllineitems}
\phantomsection\label{dessn.simple:dessn.simple.exampleIntegral.ExampleIntegral.get_prior}\pysiglinewithargsret{\bfcode{get\_prior}}{\emph{theta}}{}
Get the log prior probability given the input.

The prior distribution is currently implemented as flat prior.
\begin{quote}\begin{description}
\item[{Parameters}] \leavevmode
\textbf{theta} : array of size 2
\begin{quote}

An array representing \([\theta_1,\theta_2]\)
\end{quote}

\item[{Returns}] \leavevmode
float
\begin{quote}

the log prior probability
\end{quote}

\end{description}\end{quote}

\end{fulllineitems}

\index{plot\_observations() (dessn.simple.exampleIntegral.ExampleIntegral method)}

\begin{fulllineitems}
\phantomsection\label{dessn.simple:dessn.simple.exampleIntegral.ExampleIntegral.plot_observations}\pysiglinewithargsret{\bfcode{plot\_observations}}{}{}
Plot the observations and observation distribution.

\end{fulllineitems}


\end{fulllineitems}



\subsubsection{Module contents}
\label{dessn.simple:module-contents}\label{dessn.simple:module-dessn.simple}\index{dessn.simple (module)}

\subsection{dessn.simulation package}
\label{dessn.simulation:dessn-simulation-package}\label{dessn.simulation::doc}

\subsubsection{Submodules}
\label{dessn.simulation:submodules}

\subsubsection{dessn.simulation.observationFactory module}
\label{dessn.simulation:dessn-simulation-observationfactory-module}\label{dessn.simulation:module-dessn.simulation.observationFactory}\index{dessn.simulation.observationFactory (module)}\index{ObservationFactory (class in dessn.simulation.observationFactory)}

\begin{fulllineitems}
\phantomsection\label{dessn.simulation:dessn.simulation.observationFactory.ObservationFactory}\pysiglinewithargsret{\strong{class }\code{dessn.simulation.observationFactory.}\bfcode{ObservationFactory}}{\emph{**kwargs}}{}
Bases: \code{object}
\index{check\_kwargs() (dessn.simulation.observationFactory.ObservationFactory method)}

\begin{fulllineitems}
\phantomsection\label{dessn.simulation:dessn.simulation.observationFactory.ObservationFactory.check_kwargs}\pysiglinewithargsret{\bfcode{check\_kwargs}}{}{}
\end{fulllineitems}

\index{get\_observations() (dessn.simulation.observationFactory.ObservationFactory method)}

\begin{fulllineitems}
\phantomsection\label{dessn.simulation:dessn.simulation.observationFactory.ObservationFactory.get_observations}\pysiglinewithargsret{\bfcode{get\_observations}}{\emph{num}}{}
Still needs massive refactoring

\end{fulllineitems}


\end{fulllineitems}



\subsubsection{dessn.simulation.simulation module}
\label{dessn.simulation:module-dessn.simulation.simulation}\label{dessn.simulation:dessn-simulation-simulation-module}\index{dessn.simulation.simulation (module)}\index{Simulation (class in dessn.simulation.simulation)}

\begin{fulllineitems}
\phantomsection\label{dessn.simulation:dessn.simulation.simulation.Simulation}\pysigline{\strong{class }\code{dessn.simulation.simulation.}\bfcode{Simulation}}
Bases: \code{object}
\index{get\_simulation() (dessn.simulation.simulation.Simulation method)}

\begin{fulllineitems}
\phantomsection\label{dessn.simulation:dessn.simulation.simulation.Simulation.get_simulation}\pysiglinewithargsret{\bfcode{get\_simulation}}{\emph{num\_trans=30}}{}
\end{fulllineitems}


\end{fulllineitems}



\subsubsection{Module contents}
\label{dessn.simulation:module-contents}\label{dessn.simulation:module-dessn.simulation}\index{dessn.simulation (module)}

\section{Module contents}
\label{dessn:module-dessn}\label{dessn:module-contents}\index{dessn (module)}

\renewcommand{\indexname}{Python Module Index}
\begin{theindex}
\def\bigletter#1{{\Large\sffamily#1}\nopagebreak\vspace{1mm}}
\bigletter{d}
\item {\texttt{dessn}}, \pageref{dessn:module-dessn}
\item {\texttt{dessn.entry}}, \pageref{dessn.entry:module-dessn.entry}
\item {\texttt{dessn.entry.sim}}, \pageref{dessn.entry:module-dessn.entry.sim}
\item {\texttt{dessn.model}}, \pageref{dessn.model:module-dessn.model}
\item {\texttt{dessn.model.abstracts}}, \pageref{dessn.model:module-dessn.model.abstracts}
\item {\texttt{dessn.model.model}}, \pageref{dessn.model:module-dessn.model.model}
\item {\texttt{dessn.model.node}}, \pageref{dessn.model:module-dessn.model.node}
\item {\texttt{dessn.model.nodes}}, \pageref{dessn.model.nodes:module-dessn.model.nodes}
\item {\texttt{dessn.model.nodes.cosmology}}, \pageref{dessn.model.nodes:module-dessn.model.nodes.cosmology}
\item {\texttt{dessn.model.nodes.supernova}}, \pageref{dessn.model.nodes:module-dessn.model.nodes.supernova}
\item {\texttt{dessn.model.nodes.typeProb}}, \pageref{dessn.model.nodes:module-dessn.model.nodes.typeProb}
\item {\texttt{dessn.simple}}, \pageref{dessn.simple:module-dessn.simple}
\item {\texttt{dessn.simple.exampleIntegral}}, \pageref{dessn.simple:module-dessn.simple.exampleIntegral}
\item {\texttt{dessn.simulation}}, \pageref{dessn.simulation:module-dessn.simulation}
\item {\texttt{dessn.simulation.observationFactory}}, \pageref{dessn.simulation:module-dessn.simulation.observationFactory}
\item {\texttt{dessn.simulation.simulation}}, \pageref{dessn.simulation:module-dessn.simulation.simulation}
\end{theindex}

\renewcommand{\indexname}{Index}
\printindex
\end{document}
