%%%%%%%%%%%%%%%%%%%%%%%%%%%%%%%%%%%%%%%%%%%%%%%%%%
% Basic setup. Most papers should leave these options alone.
\documentclass[a4paper,fleqn,usenatbib]{mnras}

% MNRAS is set in Times font. If you don't have this installed (most LaTeX
% installations will be fine) or prefer the old Computer Modern fonts, comment
% out the following line
\usepackage{newtxtext,newtxmath}
% Depending on your LaTeX fonts installation, you might get better results with one of these:
%\usepackage{mathptmx}
%\usepackage{txfonts}

% Use vector fonts, so it zooms properly in on-screen viewing software
% Don't change these lines unless you know what you are doing
\usepackage[T1]{fontenc}
\usepackage{ae,aecompl}


%%%%% AUTHORS - PLACE YOUR OWN PACKAGES HERE %%%%%

% Only include extra packages if you really need them. Common packages are:
\usepackage{graphicx}	% Including figure files
\usepackage{amsmath}	% Advanced maths commands
\usepackage{amssymb}	% Extra maths symbols

%%%%%%%%%%%%%%%%%%%%%%%%%%%%%%%%%%%%%%%%%%%%%%%%%%

%%%%% AUTHORS - PLACE YOUR OWN COMMANDS HERE %%%%%

% Please keep new commands to a minimum, and use \newcommand not \def to avoid
% overwriting existing commands. Example:
%\newcommand{\pcm}{\,cm$^{-2}$}	% per cm-squared
\newcommand{\name}{SN-BHM}
\newcommand{\myemail}{samuelreay@gmail.com}
\newcommand\abs[1]{\left|#1\right|}
\newcommand {\etal} {\emph{~et~al.} }
\newcommand{\cov}{\mathcal{C}^{-1}}
\newcommand{\green}{\color{green}}
\newcommand{\blue}{\color{blue}}
\newcommand{\red}{\color{red}}

\newcommand{\kmsmpc}{{\rm km}\,{\rm s}^{-1}\,{\rm Mpc}^{-1}}

\defcitealias{Guy2010}{G10}
\defcitealias{Chotard2011}{C11}
\defcitealias{Rubin2015}{R15}	 
\newcommand{\gten}{\citetalias{Guy2010}}
\newcommand{\celeven}{\citetalias{Chotard2011}}
\newcommand{\rubin}{\citetalias{Rubin2015}}
%%%%%%%%%%%%%%%%%%%%%%%%%%%%%%%%%%%%%%%%%%%%%%%%%%

%%%%%%%%%%%%%%%%%%% TITLE PAGE %%%%%%%%%%%%%%%%%%%

% Title of the paper, and the short title which is used in the headers.
% Keep the title short and informative.
\title[\name]{\name: A hierarchical Bayesian model for Supernova Cosmology}

% The list of authors, and the short list which is used in the headers.
% If you need two or more lines of authors, add an extra line using \newauthor
\author[S. R. Hinton et al.]{
	Samuel R. Hinton,$^{1,2}$\thanks{E-mail: samuelreay@gmail.com}
	Alex G. Kim,$^{3}$
	Tamara M. Davis$^{1,2}$
	\\
	% List of institutions
	$^{1}$School of Mathematics and Physics, The University of Queensland, Brisbane, QLD 4072, Australia\\
	$^{2}$ARC Centre of Excellence for All-sky Astrophysics (CAASTRO)\\
	$^{3}$Physics Division, Lawrence Berkeley National Laboratory, 1 Cyclotron Road, Berkeley, CA 94720, USA
}

% These dates will be filled out by the publisher
\date{Accepted XXX. Received YYY; in original form ZZZ}

% Enter the current year, for the copyright statements etc.
\pubyear{2017}

% Don't change these lines
\begin{document}
\label{firstpage}
\pagerange{\pageref{firstpage}--\pageref{lastpage}}
\maketitle











% Abstract of the paper
\begin{abstract}
Abstract Abstract Abstract Abstract Abstract Abstract Abstract Abstract Abstract Abstract
Abstract Abstract Abstract Abstract Abstract Abstract Abstract Abstract Abstract Abstract
Abstract Abstract Abstract Abstract Abstract Abstract Abstract Abstract Abstract Abstract
Abstract Abstract Abstract Abstract Abstract Abstract Abstract Abstract Abstract Abstract
Abstract Abstract Abstract Abstract Abstract Abstract Abstract Abstract Abstract Abstract
Abstract Abstract Abstract Abstract Abstract Abstract Abstract Abstract Abstract Abstract
\end{abstract}

% Select between one and six entries from the list of approved keywords.
% Don't make up new ones.
\begin{keywords}
keyword1 -- keyword2 -- keyword3
\end{keywords}









%%%%%%%%%%%%%%%%% BODY OF PAPER %%%%%%%%%%%%%%%%%%

\section{Introduction}

Almost two decades have passed since the discovery of the accelerating universe \citep{Riess1998, Perlmutter1999}. Since that time, of the number of observed Type Ia supernovae (SN Ia) have increased by more than an order of magnitude thanks to modern surveys at both low redshift \citep{Bailey2008, Freedman2009, Hicken2009,  Contreras2010, Conley2011}, and higher redshift \citep{Astier2006, Wood-Vasey2007, Balland2009, Amanullah2010}. Cosmological analysis of these supernova samples \citep{Kowalski2008, Conley2011, Suzuki2012, Betoule2014, Rest2014} have been combined with complimentary probes of large scale structure \citep{Alam2017} and the CMB \citep{Hinshaw2013, PlanckCollaboration2013}, and yet, despite these prodigious efforts, the nature of dark energy remains an unsolved mystery.

In attempts to tease out the nature of dark energy, currently running and planned surveys are once again ramping up their statistical power. The Dark Energy Survey \citep[DES,][]{Bernstein2012, Abbott2016} will be observing thousands of Type Ia supernova, attaining both spectroscopic and photometric confirmation. The Large Synoptic Survey Telescope \citep[LSST,][]{Ivezic2008, LSSTScienceCollaboration2009} will produce scores of thousands of photometrically classified supernovae. Such increased statistical power demands a similarly increased fidelity and flexibility in modelling the supernovae for cosmological purposes, as systematic uncertainty will prove to be the limiting factor in our analyses.

As such, staggering effort is being put into developing more sophisticated supernovae analyses. \citet{Scolnic2016} and \citet{Kessler2017} explore sophisticated simulation corrections to traditional analyses. Approximate Bayesian computation methods also make use of simulations, trading traditional likelihoods and analytic approximations for more robust models with only the cost of increased computational time \citep{Weyant2013, Jennings2016}. Hierarchical Bayesian Models abound \citep{Mandel2009, March2011, March2014, Rubin2015, Shariff2016, Roberts2017}, however often face difficulties finding sufficient analytic approximations for complicated effects such as Malmquist bias.


In this paper, we lay out a new hierarchical model that extends on past work. Section \ref{sec:review} is dedicated to a quick review of the supernovae cosmology. In Section \ref{sec:method} we outline our methodology and apply it to simulated datasets. Forecasts for the impending DES three year spectroscopic supernova survey are contain in Section \ref{sec:des}. Section \ref{sec:sys} investigates the effect of various systematics on our model, and Section \ref{sec:details} provides details on potential areas of improvement and unsuccessful methodologies.









\section{Review}
\label{sec:review}

Whilst supernova observations take the form of time-series photometric measurements of brightness in many photometric bands, most analyses do not work from these measurements of apparent magnitude and colour. Instead, most techniques fit these observations of magnitude (along with redshift) to a supernova model, with the most widely used being that of the empirical SALT2 model \citep{Guy2007, Guy2010}. This model is trained separately before fitting the supernovae light curves for the cosmology selected supernova sample \citep{Guy2010, Mosher2014}. The resulting output from the model is, for each supernova, a characterised amplitude $x_0$ (which can be converted into apparent magnitude $m_B = -2.5\log(x_0)$), a stretch term $x_1$ and colour term $c$, along with a covariance matrix describing the uncertainty on these summary statistics. As such, the product at the end is a (redshift dependent) population of $m_B$, $x_1$ and $c$.

The underlying actual supernova population is not as clear cut, and indeed accurately characterising this population, its evolution over redshift and effects from environment is one of the challenges of supernova cosmology. However, given some modelled underlying population that lives in the redshift dependent space $M_B$, $x_1$ and $c$, the introduction of cosmology into the model is simple -- it is encoded in the functional map between those two populations, from apparent magnitude space to absolute magnitude. Specifically, for any given supernova our functional map may take the traditional form:
\begin{equation}
M_B = m_B + \alpha x_1 - \beta c - \mu(z) + \text{corrections},
\end{equation}
where $\alpha$ is the stretch correction \citep{Phillips1993}, and $\beta$ is the colour correction \citep{Tripp1998} that respectively encapsulate the empirical relation that broader and bluer supernovae are brighter. The $\text{corrections}$ term at the end often includes corrections for host galaxy environment, as this has statistically significant effects on supernova properties \citep{Kelly2010, Lampeitl2010, Sullivan2010, DAndrea2011, Gupta2011, Johansson2013, Rigault2013, Uddin2017}. The cosmological term, $\mu(z)$ represents the distance modulus, and is precisely known given cosmological parameters and an input redshift.

\subsection{Traditional Analyses}

Traditional $\chi^2$ analyses such as that found in \citet{Kowalski2008, Conley2011, Betoule2014}, minimise the difference in distance modulus between the cosmologically predicted values $\mu_C$ and the observed distance modulus $\mu_{\rm obs}$, shown respectively below:
\begin{align}
\mu_C &= 5 \log\left[ \frac{(1+z)r}{10} \right]\\
r& =\frac{c}{H_0} \int_0^z \frac{dz'}{\sqrt{ \Omega_m (1+z')^3 + \Omega_k (1 + z')^2 + \Omega_\Lambda (1+z')^{3(1+w)}}} \\
\mu_{\rm obs} &= m_B + \alpha x_1 - \beta c - M_B
\end{align}
The minimising function is then given as
\begin{equation}
\chi^2 = (\mu_{\rm obs} - \mu_C)^\dagger C^{-1} (\mu_{\rm obs} - \mu_C)
\end{equation}
where $C^{-1}$ is an uncertainty matrix which combined the uncertainty from the SALT2 fits, intrinsic dispersion, calibration, dust, peculiar velocity and many other factors (see \citet{Betoule2014} for a review). The benefit this analysis methodology provides is speed - for samples of hundreds of supernova, efficient matrix inversion algorithms allow the likelihood to be evaluated quickly. The speed comes with two costs. Firstly, formulating a $\chi^2$ likelihood requires a loss of model flexibility by building into the model assumptions of uncertainty Gaussianity. Secondly, the computational efficiency is dependent on inverting a covariance matrix with dimensionality linearly proportional to the number of supernovae. As this number increases, the cost of inversion rises quickly, and is not viable for samples with thousands of supernovae.

Selection efficiency, such as the well known Malmquist bias \citep{MalmquistK.G.1922} is accounted for by correcting data. Simulations following survey observational strategies and geometry and used to calculate the expected bias in distance modulus, which is then added onto the observational data.

\subsection{Approximate Bayesian Computation}

To try and escape the limitations of the traditional analysis methodology, several recent methods have adopted Approximate Bayesian Computation, where supernova samples are forward modelled in parameter space and compared to observed distributions. \citet{Weyant2013} provides an introduction into ABC methods for supernova cosmology in the context of the SDSS-II results \citep{Sako2014} and Flat $\Lambda$CDM cosmology, whilst \citet{Jennings2016} demonstrates their \textit{superABC} method on simulated first season Dark Energy Survey samples, described in \citet{Kessler2015}. In both examples, the supernova simulation package SNANA \citep{Kessler2009a} is used to forward model the data at each point in parameter space.

By building the systematic uncertainties and selection effects into the simulation package, there is vastly more freedom in how to treat and model those effects. Data does not need to be corrected, analytic approximations do not need to be applied, we are free to incorporate algorithms which simply cannot be expressed in a tractable likelihood. This freedom comes with a cost -- computation. The classical $\chi^2$ method's most computationally expensive step in a fit is matrix inversion. For ABC methods, we must instead simulate an entire supernova population in its entirety -- drawing from underlying supernova populations, modelling light curves, applying selection effects, fitting light curves and applying data cuts. This is an intensive process. Luckily, efficient sampling algorithms that have walkers which rely on the Markov properties of groups instead of individual walkers, such as Ensemble sampling \citep{Foreman-Mackey2013} allow easy parallelisation of parameter fits, such as used in the BAMBIS framework {\red CITE RACHEL}.

One final benefit of ABC methods is that they can move past the traditional treatment of supernovae with summary statistics ($m_B$, $x_1$ and $c$). \citet{Jennings2016} presents both a metric used to compare forward modelled summary statistic populations (denoted the `Tripp' metric) and a metric directly applicable to the observed supernova light curves themselves, however evaluation of systematic uncertainty was only performed using the Tripp metric.

\subsection{Hierarchical Bayesian Models}

Sitting comfortably between the traditional models simplicity and the complexity of forward modelling lies Hierarchical Bayesian Models. With the introduction of multiple layers in our model, we can add far more flexibility than a traditional analysis whilst still maintaining most of the computational benefits that come from having a tractable likelihood. \citet{Mandel2009} and \citet{Mandel2011} construct a hierarchical model which they apply to the light curve fitting for supernova. \citet{March2011, March2014, Karpenka2015} derive a hierarchical model and simplify it by analytically marginalising over nuisance parameters provide a model which offers increased flexibility with reduced uncertainty over the traditional method. The recent BAHAMAS model \citep{Shariff2016} builds off this and reanalyses the JLA dataset, whilst including extra freedom in the correction factors $\alpha$ and $\beta$, finding evidence for redshift dependence on $\beta$. \citet{Ma2016} also performed a reanalysis of the JLA dataset, finding significant difference in $\alpha$ and $\beta$ values from the original as well. Notably, these methods rely on data that is bias corrected, however the UNITY framework given by \citet{Rubin2015} incorporates selection efficiency analytically in the model, and is applied to the Union 2.1 dataset \citep{Suzuki2012}. The well known BEAMS (Bayesian estimation applied to multiple species) methodology from \citet{Kunz2007} has been extended and applied in several works \citep{Hlozek2012}, mostly lately to include redshift uncertainty for photometric redshift application as zBEAMS \citep{Roberts2017}. 

The flexibility afforded by a hierarchical model allows for investigations into different treatments of underlying populations, rates, redshift distributions, mass corrections and redshift evolution, each of will will be discussed further in the outline of our model below.









\section{Our Method}
\label{sec:method}

We construct our Bayesian Hierarchical Model with several goals in mind: creation of a redshift dependent correlated underlying supernova population, increased treatment of systematics, and analytic correction of selection effects. As this is closest to the UNITY method from \citet[][hereafter denoted \rubin]{Rubin2015}, we follow a similar model scaffold, and construct the model in Stan \citep{Carpenter2017, StanDevelopmentTeam2017} which uses automatic differentiation and the no-U-turn Sampler (NUTS), which is a variant of Hamiltonian Monte Carlo, to efficiently sample high dimensional parameter space.

At the most fundamental level, a supernova analysis is simply a mapping from an underlying population onto an observed population, where cosmology is encoded directly in the mapping function. The difficulty arises in adding sufficient, physically motivated flexibility in both of these populations whilst not adding \textit{too} much flexibility, such that model fitting becomes pathological due to increasing parameter degeneracies within the model.

\subsection{Observed and Latent Population}

Like most of the BHM methods introduced previously, we work from the summary statistics as well, where each observed supernova has an apparent magnitude $\hat{m_B}$, stretch $\hat{x_1}$ and colour $\hat{c}$, with uncertainty $\cov$ on those values. Additionally, each supernova has an observed redshift $\hat{z}$ and a host galaxy mass associated with it, $\hat{m}$. Given machine learning classifiers to type supernovae, we will also have a probability of being a Type Ia, $\hat{p}$.

The first layer of the hierarchy represents the latent population of $m_B$, $x_1$ and $c$. This can be thought of as the `true' values for the supernova, and is normally distributed around the observed values such that 
\begin{equation}
P(m_B, x_1, c|\hat{m_B},\hat{x_1},\hat{c},\cov) = \mathcal{N}(\lbrace m_B, x_1, c \rbrace|\lbrace \hat{m_B},\hat{x_1},\hat{c} \rbrace, \cov).
\end{equation}
As we are focused on the spectroscopically confirmed DES supernovae for this iteration of the method, we assume the observed redshift $\hat{z}$ is the true redshift $z$, and the similarly that the observed mass is the true mass $m$. Whilst the latter is not physically motivated like the redshift measurement, the relatively small contribution of the host galaxy mass to the model makes this an acceptable approximation.

\subsection{Underlying Population}

The underlying supernova population is often treated with two components - a population distribution in colour and stretch, and intrinsic dispersion. Analytic models often treat the colour and stretch populations with a skew normal and normal, respectively, and have them as independent. Intrinsic dispersion is treated in a variety of manners in other models, from representing it simply as (Gaussian) scatter on the absolute magnitude of the supernova population, to correlated multivariate normal scatter on the combined magnitude, stretch and colour distribution. Additionally, redshift drift of populations' mean colour and stretch will introduce a cosmological bias in our fits unless the population possesses similar ability to change as a function of redshift. 

We model the underlying population and intrinsic dispersion together as a redshift dependent multivariate skew normal for each survey. Following {\rubin} we allow the mean colour and stretch to vary over redshift, anchoring four equally spaced redshift nodes spanning the redshift range of each survey, linearly interpolating between the nodes. Both the colour and stretch means are modelled with normal priors. The correlation between the magnitude, stretch and colour populations is represented in a correlation matrix $\rho$, which is treated with an LKJ prior. Skewness is fixed to 0 for the absolute magnitude population, but left free for colour and stretch, with normal priors centered at 0 to draw the skewness to zero if not well constrained by the data. The decision to include skewness on the stretch distribution as well as the colour was motivated by \citet{Scolnic2016}, which found highly asymmetric underlying populations for both colour and stretch, regardless of survey or scatter model adopted.

The width of the population, represented by the vector $\lbrace \sigma_{M_B}, \sigma_{x_1}, \sigma_c \rbrace$ is subject to Cauchy priors, however are sampled in log space for efficiency in sampling close to zero. 


As such, the only constant between survey populations is the absolute magnitude $M_B$, with the skewness, redshift dependent means, width and correlations fit individual for each survey.

On top of the Ia populations, as described above, we also include a simplistic outlier population that also follows {\rubin} (and therefore \citet{Kunz2007}) as a Gaussian mixture; where the mean of the population is fixed to the Ia population, but the population width is set to a width of $\sigma^{\rm outl} = 1$ in $M_B$, $x_1$ and $c$. With the spectroscopic DES sample, the contamination rate is expected to be far too low to actually fit contamination population, however in future works with photometric samples which will suffer from significantly more contamination it will be required that extra degrees of freedom are afforded the outlier population. Proof of concept simulation fits show that an acceptable parameterisation is to represent the typically brighter contaminant population as $\langle M_B^{\rm outl} \rangle = \langle M_B \rangle - \delta_{M_B}^{\rm outl}$, where $\delta_{M_B}^{\rm outl}$ is constrained to be positive, or even to be greater than a small positive number to reduce degeneracy between the two populations. For the purposes of the DES spectroscopic sample which will be dominated by confirmed Type Ia supernovae, $\delta_{M_B}^{\rm outl} =0$. We assume that supernova fall into either population as determined by their observed classification probability $\hat{p}$.


\subsection{Population Map}

\subsubsection{Cosmology}

We formulate our model with three different cosmological parameterisations; Flat $\Lambda$CDM, Flat $w$CDM and standard $\Lambda$CDM. $\Omega_m$ is given the prior $\mathcal{U}(0.05, 0.99)$, $\Omega_\Lambda$ was treated with $\mathcal{U}(0, 1.5)$ and the equation of state $w$ was similarly set to a flat prior $\mathcal{U}(-0.4, -2.0)$. For calculating the distance modulus, we fix $H_0 = 70 \kmsmpc $. 

\subsubsection{Standardisation Parameters}

With increasingly large datasets and more nuanced analyses, the choice of how to handle $\alpha$ and $\beta$ becomes an important consideration when constructing a model. {\rubin} employs a broken linear relationship for both colour and stretch, where different values of $\alpha$ and $\beta$ are adopted depending on whether $x_1$ and $c$ are respectively positive or negative (although the cut could be placed at a location other than 0). \citet{Shariff2016} instead of employing a colour-dependent $\beta$, model $\beta$ as redshift dependent, testing two phenomenological models; $\beta(z) = \beta_0 + \beta_1 z$ and $\beta(z) = \beta_0 + \Delta \beta\left(0.5 + \arctan(100(z - z_t)) / \pi\right)$, where the later effects a rapid but smooth change in $\beta$ at a turnover redshift $z_t$.

We tested two models against simulated supernova sets; $\beta(c) = \beta_0 + \beta_1 c$ and $\beta(z) = \beta_0 + \beta_1 z$. See Section \ref{sec:simdes} for details on simulation generation. We found for both models that non-zero values for $\beta$ are preferred (even with constant $\beta$ used in simulation) due to severe degeneracy with selection effects. This degeneracy resulted in a significant bias in cosmology, and so in our final model we continue to adopt the constant $\alpha$ and $\beta$ found in traditional analyses.

\subsubsection{Host Galaxy Environment}

It is now well known that host galaxy environment has a significant effect on supernova properties. The latest sample of over 1300 spectroscopically confirmed Type Ia supernova show $>5\sigma$ evidence for correlation between host mass and luminosity \citep{Uddin2017}. The traditional correction, as employed in analyses such as \citet{Suzuki2012} and \citet{Betoule2014} invoke a step function such that $\Delta M = 0.08 \mathcal{H}(log(M) - 10))$, where $\mathcal{H}$ is the Heaviside step function and $M$ is the galaxy mass in solar masses. The scale of this step function varies from analysis to analysis, with the 0.08 value shown previously sourced from \cite{Sullivan2010} and used in \citet{Betoule2014}. In this work we adopt the model used in {\rubin}, which follows the work from \citet{Rigault2013}, such that we introduce two parameters to incorporate a redshift-dependent host galaxy mass correction:
\begin{equation}
\Delta M = \delta(0) \left[ \frac{1.9\left(1 - \frac{\delta(0)}{\delta(\infty)}\right)  }{0.9 + 10^{0.95z}} + \frac{\delta(0)}{\delta(\infty)}\right]
\end{equation}
We also take flat priors on the parameterisation $\delta(0)$, $\delta(0)/\delta(\infty)$.


\subsubsection{Selection Effects}

\subsubsection{Systematics}



\vspace{20mm}
Mapping population of observables on a population of underlying SN, where the map function encodes
cosmology. Difficulty is creating an underlying SN population that is flexible enough to not introduce bias whilst still being physically motivated. 


Observables -> Transformation function (latent, mass, cosmology, systematics) -> Underlying pop (and outlier)

\subsection{Applied to Spectroscopic Sample}

Minimal outliers

\section{Application to DES}
\label{sec:des}

\subsection{Simulating DES SN data}
\label{sec:simdes}
{\green BUT THE BELOW IN THE TESTING AREA}

Early analyses often treated intrinsic dispersion simply as scatter in the underlying absolute magnitude of the underlying population, but recent analyses require more a more sophisticated approach. In our development of this model and tests of intrinsic dispersion, we analyse the effects of two different scatter models. The first is the model from \citet[][hereafter denoted the {\gten} scatter model]{Guy2010}, which models intrinsic scatter with a 70\% contribution from coherent variation and 30\% from chromatic variation. The second model, denoted the {\celeven} model is sourced from \citet{Chotard2011} and has variation with 25\% contribution from coherent scatter and 75\% from chromatic variation.

\subsection{Model validation}

appoximate\_simple\_test.py

multisim

bulk


\subsection{Results on simulated data (ie projections)}

\subsubsection{Spectroscopic Sample}

\subsubsection{Photometric sample}

\subsection{Comparison with bells and whistles fixed}


\section{Systematics Strength Test}
\label{sec:sys}

systematics test

\section{Interesting Implementation Details}
\label{sec:details}
Anything interesting.

Also talk about non-analytic correction factors (and their failure - mc integration, GP, NNGP)



\section{Conclusions}



\section*{Acknowledgements}





%%%%%%%%%%%%%%%%%%%% REFERENCES %%%%%%%%%%%%%%%%%%

% The best way to enter references is to use BibTeX:

\bibliographystyle{mnras}
\bibliography{bib}




%%%%%%%%%%%%%%%%% APPENDICES %%%%%%%%%%%%%%%%%%%%%

\appendix

\section{Papers}




%%%%%%%%%%%%%%%%%%%%%%%%%%%%%%%%%%%%%%%%%%%%%%%%%%


% Don't change these lines
\bsp	% typesetting comment
\label{lastpage}
\end{document}

% End of mnras_template.tex