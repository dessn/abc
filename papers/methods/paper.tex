%%%%%%%%%%%%%%%%%%%%%%%%%%%%%%%%%%%%%%%%%%%%%%%%%%
% Basic setup. Most papers should leave these options alone.
\documentclass[a4paper,fleqn,usenatbib]{mnras}

% MNRAS is set in Times font. If you don't have this installed (most LaTeX
% installations will be fine) or prefer the old Computer Modern fonts, comment
% out the following line
\usepackage{newtxtext,newtxmath}
% Depending on your LaTeX fonts installation, you might get better results with one of these:
%\usepackage{mathptmx}
%\usepackage{txfonts}

% Use vector fonts, so it zooms properly in on-screen viewing software
% Don't change these lines unless you know what you are doing
\usepackage[T1]{fontenc}
\usepackage{ae,aecompl}


%%%%% AUTHORS - PLACE YOUR OWN PACKAGES HERE %%%%%

% Only include extra packages if you really need them. Common packages are:
\usepackage{graphicx}	% Including figure files
\usepackage{amsmath}	% Advanced maths commands
\usepackage{amssymb}	% Extra maths symbols

%%%%%%%%%%%%%%%%%%%%%%%%%%%%%%%%%%%%%%%%%%%%%%%%%%

%%%%% AUTHORS - PLACE YOUR OWN COMMANDS HERE %%%%%

% Please keep new commands to a minimum, and use \newcommand not \def to avoid
% overwriting existing commands. Example:
%\newcommand{\pcm}{\,cm$^{-2}$}	% per cm-squared
\newcommand{\name}{SN-BHM}
%%%%%%%%%%%%%%%%%%%%%%%%%%%%%%%%%%%%%%%%%%%%%%%%%%

%%%%%%%%%%%%%%%%%%% TITLE PAGE %%%%%%%%%%%%%%%%%%%

% Title of the paper, and the short title which is used in the headers.
% Keep the title short and informative.
\title[\name]{\name: A hierarchical Bayesian model for Supernova Cosmology}

% The list of authors, and the short list which is used in the headers.
% If you need two or more lines of authors, add an extra line using \newauthor
\author[S. R. Hinton et al.]{
	Samuel R. Hinton,$^{1,2}$\thanks{E-mail: samuelreay@gmail.com}
	Alex G. Kim,$^{3}$
	Tamara M. Davis$^{1,2}$
	\\
	% List of institutions
	$^{1}$School of Mathematics and Physics, The University of Queensland, Brisbane, QLD 4072, Australia\\
	$^{2}$ARC Centre of Excellence for All-sky Astrophysics (CAASTRO)\\
	$^{3}$Physics Division, Lawrence Berkeley National Laboratory, 1 Cyclotron Road, Berkeley, CA 94720, USA
}

% These dates will be filled out by the publisher
\date{Accepted XXX. Received YYY; in original form ZZZ}

% Enter the current year, for the copyright statements etc.
\pubyear{2017}

% Don't change these lines
\begin{document}
\label{firstpage}
\pagerange{\pageref{firstpage}--\pageref{lastpage}}
\maketitle











% Abstract of the paper
\begin{abstract}
Abstract Abstract Abstract Abstract Abstract Abstract Abstract Abstract Abstract Abstract
Abstract Abstract Abstract Abstract Abstract Abstract Abstract Abstract Abstract Abstract
Abstract Abstract Abstract Abstract Abstract Abstract Abstract Abstract Abstract Abstract
Abstract Abstract Abstract Abstract Abstract Abstract Abstract Abstract Abstract Abstract
Abstract Abstract Abstract Abstract Abstract Abstract Abstract Abstract Abstract Abstract
Abstract Abstract Abstract Abstract Abstract Abstract Abstract Abstract Abstract Abstract
\end{abstract}

% Select between one and six entries from the list of approved keywords.
% Don't make up new ones.
\begin{keywords}
keyword1 -- keyword2 -- keyword3
\end{keywords}









%%%%%%%%%%%%%%%%% BODY OF PAPER %%%%%%%%%%%%%%%%%%

\section{Introduction}

Introduce significance of SN historically, and the quest to remain competitive with BAO and other probes, we have to reduce systematic uncert whilst larger surveys reduce stat. 

Review old methods

Review current methods

Potential places for improvement

\section{Our Method}

General comments about the method (BHM), Stan

\subsection{General Description}

Mapping population of observables on a population of underlying SN, where the map function encodes
cosmology. Difficulty is creating an underlying SN population that is flexible enough to not introduce bias whilst still being physically motivated. 


Observables -> Transformation function (latent, mass, cosmology, systematics) -> Underlying pop (and outlier)

\subsection{Applied to Spectroscopic Sample}

Minimal outliers

\section{Application to DES}

\subsection{Simulating DES SN data}

\subsection{Results on simulated data (ie projections)}

\subsubsection{Spectroscopic Sample}

\subsubsection{Photometric sample}

\subsection{Comparison with bells and whistles fixed}

\subsection{Model validation}

appoximate\_simple\_test.py

multisim

bulk

\section{Systematics Strength Test}

systematics test

\section{Interesting Implementation Details}

Anything interesting.

Also talk about non-analytic correction factors (and their failure - mc integration, GP, NNGP)



\section{Conclusions}



\section*{Acknowledgements}





%%%%%%%%%%%%%%%%%%%% REFERENCES %%%%%%%%%%%%%%%%%%

% The best way to enter references is to use BibTeX:

\bibliographystyle{mnras}
\bibliography{bib}




%%%%%%%%%%%%%%%%% APPENDICES %%%%%%%%%%%%%%%%%%%%%

\appendix

\section{Papers}




%%%%%%%%%%%%%%%%%%%%%%%%%%%%%%%%%%%%%%%%%%%%%%%%%%


% Don't change these lines
\bsp	% typesetting comment
\label{lastpage}
\end{document}

% End of mnras_template.tex